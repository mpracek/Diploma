\documentclass[12pt,a4paper]{amsart}
% ukazi za delo s slovenscino -- izberi kodiranje, ki ti ustreza
\usepackage[slovene]{babel}
\usepackage[cp1250]{inputenc}
%\usepackage[T1]{fontenc}
%\usepackage[utf8]{inputenc}
\usepackage{amsmath,amssymb,amsfonts}
\usepackage{url}
%\usepackage[normalem]{ulem}
\usepackage[dvipsnames,usenames]{color}

% ne spreminjaj podatkov, ki vplivajo na obliko strani
\textwidth 15cm
\textheight 24cm
\oddsidemargin.5cm
\evensidemargin.5cm
\topmargin-5mm
\addtolength{\footskip}{10pt}
\pagestyle{plain}
\overfullrule=15pt % oznaci predlogo vrstico


% ukazi za matematicna okolja
\theoremstyle{definition} % tekst napisan pokoncno
\newtheorem{definicija}{Definicija}[section]
\newtheorem{primer}[definicija]{Primer}
\newtheorem{opomba}[definicija]{Opomba}

\renewcommand\endprimer{\hfill$\diamondsuit$}


\theoremstyle{plain} % tekst napisan posevno
\newtheorem{lema}[definicija]{Lema}
\newtheorem{izrek}[definicija]{Izrek}
\newtheorem{trditev}[definicija]{Trditev}
\newtheorem{posledica}[definicija]{Posledica}


% za stevilske mnozice uporabi naslednje simbole
\newcommand{\R}{\mathbb R}
\newcommand{\N}{\mathbb N}
\newcommand{\Z}{\mathbb Z}
\newcommand{\C}{\mathbb C}
\newcommand{\Q}{\mathbb Q}


% ukaz za slovarsko geslo
\newlength{\odstavek}
\setlength{\odstavek}{\parindent}
\newcommand{\geslo}[2]{\noindent\textbf{#1}\hspace*{3mm}\hangindent=\parindent\hangafter=1 #2}


% naslednje ukaze ustrezno popravi
\newcommand{\program}{Finan�na matematika} % ime studijskega programa: Matematika/Finan"cna matematika
\newcommand{\imeavtorja}{Martin Pra�ek} % ime avtorja
\newcommand{\imementorja}{prof.~dr./doc.~dr. Damjan �kulj} % akademski naziv in ime mentorja
\newcommand{\naslovdela}{Skriti markovski modeli v �asovnih vrstah}
\newcommand{\letnica}{2019} %letnica diplome


% vstavi svoje definicije ...




\begin{document}

% od tod do povzetka ne spreminjaj nicesar
\thispagestyle{empty}
\noindent{\large
UNIVERZA V LJUBLJANI\\[1mm]
FAKULTETA ZA MATEMATIKO IN FIZIKO\\[5mm]
\program\ -- 1.~stopnja}
\vfill

\begin{center}{\large
\imeavtorja\\[2mm]
{\bf \naslovdela}\\[10mm]
Delo diplomskega seminarja\\[1cm]
Mentor: \imementorja}
\end{center}
\vfill

\noindent{\large
Ljubljana, \letnica}
\pagebreak

\thispagestyle{empty}
\tableofcontents
\pagebreak

\thispagestyle{empty}
\begin{center}
{\bf \naslovdela}\\[3mm]
{\sc Povzetek}
\end{center}
% tekst povzetka v slovenscini
V povzetku na kratko opi"si vsebinske rezultate dela. Sem ne sodi razlaga organizacije dela -- v katerem poglavju/razdelku je kaj, pa"c pa le opis vsebine.
\vfill
\begin{center}
{\bf Hidden Markov Models in Time Series}\\[3mm] % prevod slovenskega naslova dela
{\sc Abstract}
\end{center}
% tekst povzetka v anglescini
Prevod zgornjega povzetka v angle"s"cino.

\vfill\noindent
{\bf Math. Subj. Class. (2010):} navedi vsaj eno klasifikacijsko oznako -- dostopne so na \url{www.ams.org/mathscinet/msc/msc2010.html}  \\[1mm]
{\bf Klju"cne besede:} skriti markovski modeli, �asovne vrste, slu�ajni proces navedi nekaj klju"cnih pojmov, ki nastopajo v delu  \\[1mm]
{\bf Keywords:}  hidden markov models, time series, angle"ski prevod klju"cnih besed
\pagebreak



% tu se zacne besedilo seminarja
\section{Uvod}
Skriti markovski modeli so �iroko uporabni 





% slovar
\section*{Slovar strokovnih izrazov}

%\geslo{}{}
%
%\geslo{}{}
%

\section{Uporaba}
Skriti markovski modeli so zelo �iroko uporabno matemati�no orodje za modeliranje. Na�ini uporabe se zelo razlikujejo in gredo od zelo biolo�kih do ekonomskih. Ekonomskim, torej tistim, ki jih delujejo kot napovedovalci cen vrednostnih papirjev v prihodnosti, se bom najglobje posvetil v tem delu, zato sedaj raje poglejmo ostale na�ine uporabe.



\subsection{Procesiranje govora}

Ena najbolj �iroko uporabljenih na�inov uporabe pa je procesiranje govora za posami�ne glasovne enote. Gre za sistem, kjer �elimo prepoznati posami�ne izgovorjene besede, ne moremo pa ga uporabiti za splo�en govor. \\
Da bomo procesiranje lahko zmodelirali moramo najprej dolo�iti slovar glasov, iz katerega vemo, da bo glas pri�el. Tu uporabljam besedo glas, ker ni nujno, da bo prepoznan glas dejansko beseda; eden izmed glasov bi lahko bil tudi zgolj �rka A. Ko je slovar dolo�en, bo slu�ajni proces $St$, kjer $t$ ozna�uje realizacijo v kon�nem �asu $t$, dolo�en z dvema mehanizmoma: 
\begin{enumerate}
	\item homogena Markova veriga $Ct$, $t\in1,..,m$, ki dolo�a polo�aj glasilk v vsakem �asu,
	\item ter verjetnostne porazdelitve, po eno za vsako stanje.
\end{enumerate}
Te pogoje natan�no dolo�imo s parametri $\delta$, ki predtavlja $C1$, $\Pi$, kjer je to matrika velikosti $n\times m$ in pa prehodno matriko $\Gamma$ za markovsko verigo.\\
Ko imamo vse to dolo�eno s pomo�jo Baum-Welchovega algoritma dolo�imo rezultat.




\subsection{Uporaba v biologiji in biokemiji}
V zadnjem �asu se skrite markovske modele vedno ve� uporablja tudi za modeliranje biolo�kih in biokemijskih procesov, med katerimi je najbolj  znan primer modeliranja proteinov v celi�ni membrani, ki ga bom malo opisal, poleg tega pa med drugim �e za {\bf napovedovanje genov?}. \\
�eprav lahko nekatere proteine in njihovo delovanje znotraj celi�ne membrane enostavno predstavimo, to ne velja za vse. Tako imenovani $\beta$
-\textbf{barrel} membranski proteini zahtevajo ve� dela. Da bi ugotovili njihovo delovanje in ga primerjali z delovanjem v vodi topnih proteinov so Bagos, Liakoupos et al. razvili model ki to naredi. Gre za model, ki temelji na skritih markovskih modelih, njegov trening pa poteka na diskretni bazi podatkov in ni namenjen maksimiziranju \textbf{likelihood}, temve� je namenjen maksimiziranju verjetnosti pravilnih napovedi.\\
Z delom potem nadaljujemo podobno kot pri procesiranju govora, le da je na� za�etni slovar tu dolg $20$ znakov, toliko kolikor je aminokislin.


% seznam uporabljene literature
\begin{thebibliography}{99}

%\bibitem{}

\end{thebibliography}

\end{document}

